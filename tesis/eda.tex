\section{Mapas de la dinámica molecular en nubes de formación estelar muy cercanas}

Actualmente no existe una definición clara de qu\'e es un clúster de estrellas. T\'ecnicas como el cálculo de la distribución de las densidades de la superficie estelar no han dado resultados diferentes para regiones de formación stelar aisladas (\emph{isolated}) y agrupadas (\emph{clustered}). Un enfoque de teoría de grafos, por otro lado, requiere de mucha precisión para lograr el objetivo de determinar qu\'e es un clúster de estrellas, teniendo el ojo humano muchos mejores resultados que los actuales algoritmos en esta materia. Tambi\'en se puede definir un clúster de estrellas basado en la energía de unión (\emph{binding energy}) de un grupo de estrellas. Esto último puede desencadenar en la obtención de asociaciones estelares, un subconjunto de un clúster de estrellas abierto que, para algunos, ya no sería un clúster de estrellas. Como se podrá observar, la definición de qu\'e es un clúster de estrellas puede ser muy difícil, y a menudo es meramente una materia de opinión personal \cite{CSFreview}. 

La función de masa inicial (\gls{imf}) tampoco entrega mucha información al respecto, mostrándose invariante tanto para densos clústers (\emph{dense clusters}) y asociaciones dispersas (\emph{sparse associations}), como para cúmulos globulares (\emph{globular clusters}) y abiertos (\emph{open clusters}) \cite{CSFreview}.

\section{Estimación de parámetros de zonas de formación estelar muy lejanas}

\gls{lh}

\gls{rs}

\gls{sfh}
